\documentclass[aspectratio=169, xetex, 12pt]{beamer}
\usepackage[utf8]{inputenc}
\usepackage[T1]{fontenc}
\usepackage{datetime}

\usetheme[mode=dark]{extia}
\title{Les nouveautés de Python 3.11}
\author{Clément Dubos}
\institute{Extia}
\newdate{date}{06}{09}{2022}
\date{\displaydate{date}}

\begin{document}

    \maketitle

    \begin{frame}{Sommaire}
        \hfill
        \parbox[t]{.9\textwidth}{
            \begin{minipage}[c][0.2\textheight]{\textwidth}
                \Large
                \tableofcontents
            \end{minipage}
        }
    \end{frame}

    \section{Optimization and Performance}

    \begin{frame}{Faster CPython}{Generalité}
        \begin{minipage}[t]{0.5\paperwidth}
            \begin{block}{Faster CPython, un projet ambitieux}
                \begin{itemize}
                    \item Initiative: Microsoft et Guido Van Rossum
                    \item Objectif: x5 en 2025 => 1.5/an
                    \item basé sur HotPy et HotPy2
                    \item Incrémental sur les différentes version de python a venir.
                \end{itemize}
            \end{block}
        \end{minipage}
        \hspace{.1cm}
        \begin{minipage}[t]{0.4\paperwidth}
            \begin{block}{Statut actuel}
                \begin{itemize}
                    \item Speedup x1.25 en moyenne
                    \item Entre x1.10 et x1.60 selon le workload
                    \item \href{https://docs.python.org/3.11/whatsnew/3.11.html\#faster-cpython}{Python 3.11 Faster Cpython}
                \end{itemize}
            \end{block}
        \end{minipage}
    \end{frame}

    \begin{frame}{Faster CPython}{Détail}

        \begin{block}{Speedup du démarrage}
            \begin{itemize}
                \item code object et bytecode statiquement alloué par l'interpréteur
            \end{itemize}
        \end{block}

        \begin{block}{Speedup du runtime}
            \begin{itemize}
                \item Reduction de l'information d'execution en memoire (debug ou on-demand). 3-7\%
                \item mechanisme de "jump" lors d'un appel de function. 1-3\%
                \item  Specializing Adaptive Interpreter
            \end{itemize}
        \end{block}
        \begin{block}{Misc}
            \begin{itemize}
                \item les objets nécessite moins d'espace mémoire
                \item Représentation des exceptions plus concise, améliore le catchin de 10\%
            \end{itemize}
        \end{block}
    \end{frame}

    \begin{frame}{Optimisation Global}
        \begin{block}{}
            \begin{itemize}
                \item C-style formatting est maintenant aussi efficace que les f-string
                \item "Zero-cost" exception => cout du "try" casi null lorsqu'il n'y a pas d'exception
                \item Les dict ne stock plus les hash lorsqu'il ne manipule que des objet unicode => reduction espace mémoire
                \item module "re" refactorisé, 10\% plus rapide.
            \end{itemize}
        \end{block}
        %https://docs.python.org/3.11/whatsnew/3.11.html#optimizations
    \end{frame}

    \section{Exceptions}
    \begin{frame}{Exception Groups and except*}
        %https://peps.python.org/pep-0654/
    \end{frame}

    \begin{frame}{Enriching Exceptions with Notes}
        %https://docs.python.org/3.11/whatsnew/3.11.html#exceptions-can-be-enriched-with-notes-pep-678
    \end{frame}

    \begin{frame}{Improved Error Locations in Tracebacks}
        %https://docs.python.org/3.11/whatsnew/3.11.html#enhanced-error-locations-in-tracebacks
    \end{frame}

    \section{TOML Parser}
    \begin{frame}{TOML Parser}
        %https://docs.python.org/3.11/whatsnew/3.11.html#new-modules
    \end{frame}

    \section{Major Language Changes}
    \begin{frame}{Major Language Changes}
        %https://docs.python.org/3.11/whatsnew/3.11.html#other-language-changes
    \end{frame}

    \section{Major Modules Updates}
    \begin{frame}{Major Modules Updates}
        \begin{minipage}{0.49\paperwidth}
            \begin{block}{}
                \begin{itemize}
                    \item AsyncIO => TaskGroup context manager
                    \item contextlib => chdir context manager
                    \item re => atomic groups et possessive quantifier
                    \item unittest => enterContext() et enterClassContext()
                \end{itemize}
            \end{block}
        \end{minipage}
        \hspace{0.01cm}
        \begin{minipage}{0.49\paperwidth}
            \begin{block}{}

            \end{block}
        \end{minipage}
    \end{frame}

    \section{Typing improvements}

    \begin{frame}{Variadic generics}
        %https://docs.python.org/3.11/whatsnew/3.11.html#new-features-related-to-type-hints
    \end{frame}

    \begin{frame}{individual TypedDict items}
        %https://docs.python.org/3.11/whatsnew/3.11.html#new-features-related-to-type-hints
    \end{frame}

    \begin{frame}{Self type}
        %https://docs.python.org/3.11/whatsnew/3.11.html#new-features-related-to-type-hints
    \end{frame}

    \begin{frame}{Arbitrary literal string type}
        %https://docs.python.org/3.11/whatsnew/3.11.html#new-features-related-to-type-hints
    \end{frame}

    \begin{frame}{Data Class Transforms}
        %https://docs.python.org/3.11/whatsnew/3.11.html#new-features-related-to-type-hints
    \end{frame}


\end{document}
